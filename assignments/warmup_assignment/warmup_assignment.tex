\documentclass[a4paper,10pt, notitlepage]{report}
\usepackage[utf8]{inputenc}
\usepackage{natbib}
\usepackage{amssymb}
\usepackage{amsmath}
\usepackage{amsthm}
\usepackage[shortlabels]{enumitem}
% \usepackage[portuguese]{babel}

%%%%%%%%%%%%%%%%%%%%%%%%%%%%%%%%%%%%%%%%%%%%%%%%
% Proper definitions
%%%%%%%%%%%%%%%%%%%%%%%%%%%%%%%%%%%%%%%%%%%%%%%%
\newcommand{\R}{\mathbb{R}}
\newcommand{\Q}{\mathbb{Q}}
\newcommand{\Z}{\mathbb{Z}}
\newcommand{\N}{\mathbb{N}}
\newcommand{\tr}{\operatorname{tr}}
\newcommand{\var}{\operatorname{Var}}
\newcommand{\unif}{\operatorname{Unif}}
\newcommand{\ev}{\mathbb{E}}
\newcommand{\pr}{\mathbb{P}}


% Title Page
\title{Assignment 0: O Brother, How Far Art Thou?}
\author{Computational Statistics \\ Instructor: Luiz Max de Carvalho}

\begin{document}
\maketitle

\textbf{Hand-in date: 06/10/2020.}

\section*{General guidance}
\begin{itemize}
 \item State and prove all non-trivial mathematical results necessary to substantiate your arguments;
 \item Do not forget to add appropriate scholarly references~\textit{at the end} of the document;
 \item Mathematical expressions also receive punctuation;
 \item Please hand in a single PDF file as your final main document.
 
 Code appendices are welcome,~\textit{in addition} to the main PDF document.
 \end{itemize}

\newpage

\section*{Background}

A large portion of the content of this course is concerned with computing high-dimensional integrals~\textit{via} simulation.
Today you will be introduced to a simple-looking problem with a complicated closed-form solution and one we can approach using simulation.

Suppose you have a disc $C_R$ of radius $R$. 
Take $p = (p_x, p_y)$ and $ q = (q_x, q_y) \in C_R$ two points in the disc.  
Consider the Euclidean distance between $p$  and $q$, $||p-q|| = \sqrt{(p_x-q_x)^2 + (p_y-q_y)^2} = |p-q|$.
\paragraph{Problem A:} What is the \textit{average} distance between pairs of points in $C_R$ if they are picked uniformly at random?

\section*{Questions}

\begin{enumerate}
 \item To start building intuition, let's solve a related but much simpler problem.
 Consider an interval $[0, s]$, with $s>0$ and take $x_1,x_2 \in [0, s]$~\textit{uniformly at random}.
 Show that the average distance between $x_1$ and $x_2$ is $s/3$.

 \begin{proof}[Solution]
    We want to calculate $\ev[|x_1 - x_2|]$ such that $x_1, x_2
    \overset{iid}{\sim} \unif(0,s)$. An extensive way to solve this problem is
    to introduce the random variable $Y = x_1 - x_2$ and derive its
    distribution and after its absolute mean. Let
    \begin{equation*}
        F_Y(y) = \pr(Y \le y) = \pr(x_1 \le y + x_2).
    \end{equation*}
    Suppose first that $y \ge 0$. Given
    $x_2$, 
    $$\pr(x_1 \le y + x_2|x_2) = \begin{cases}
        \dfrac{y + x_2}{s} &0 \le x_2 \le s - y \\
        1 &s - y < x_2 \le s.
    \end{cases}$$
    Therefore, 
    \begin{equation*}
        \begin{split}
            F_Y(y) &= \int_0^{s-y} \frac{y+x_2}{s^2} \, dx_2 + \int_{s-y}^s \frac{1}{s} \, dx_2  \\
            &= \frac{y(s-y) + 0.5(s-y)^2}{s^2} + \frac{y}{s} \\
            &= \frac{(s-y)(y + s)}{2s^2} + \frac{y}{s} = - \frac{y^2}{2s^2} + \frac{y}{s} + \frac{1}{2}.
        \end{split}
    \end{equation*}
    Now suppose $y < 0$. Then 
    $$\pr(x_1 \le y + x_2|x_2) = \begin{cases}
        \dfrac{y + x_2}{s} &-y \le x_2 \le s \\
        0 &0 \le x_2 < -y.
    \end{cases}$$
    Therefore, 
    \begin{equation*}
        \begin{split}
            F_Y(y) &= \int_{-y}^{s} \frac{y+x_2}{s^2} \, dx_2 \\
            &= \frac{ys + 0.5s^2}{s^2} - \frac{-y^2 + 0.5y^2}{s^2} \\
            &= \frac{y^2}{2s^2} + \frac{y}{s} + \frac{1}{2}.
        \end{split}
    \end{equation*}
    Deriving these expressions we get the density with respect to the Lebesgue
    measure, 
    \begin{equation*}
        f_Y(y) = \begin{cases}
            \dfrac{1}{s} - \dfrac{y}{s^2}, &\text{ if } 0 \le y \le s \\
            \dfrac{1}{s} + \dfrac{y}{s^2}, &\text{ if } -s \le y < 0. \\
        \end{cases}
    \end{equation*}
    Finally, 
    \begin{equation*}
        \begin{split}
            \ev[|x_1 - x_2|] = \ev[|Y|] &= \int_0^s \frac{y}{s} - \frac{y^2}{s^2} \, dy - \int_{-s}^0 \frac{y}{s} + \frac{y^2}{s^2} \, dy \\
            &= \frac{s^2}{2s} - \frac{s^3}{3s^2} + \frac{s^2}{2s} - \frac{s^3}{3s^2} \\
            &= \frac{s}{2} - \frac{s}{3} + \frac{s}{2} - \frac{s}{3} = \frac{s}{3},
        \end{split}
    \end{equation*}
    as we wished to prove. Now I give a more geometric approach. Consider
    $(x_1, x_2)$ uniformly distributed over $[0,1]^2$. Then, let $t \in [0,s]$, 
    $$
    \pr(|x_1 - x_2| > t) = \pr(x_1 > t + x_2) + \pr(x_2 > t + x_1),
    $$
    since they are disjoint events, since $t \ge 0$. Note that the 
    first region is delimited by the straight line $x_1 = t + x_2$, $x_1 = s$,
    and $x_2 = 0$, what gives a triangle with points $(t,0), (s, 0), $ and
    $(s,s-t)$
    with area $(s-t)^2/2$. The second region is delimited by the straight line
    $x_2 = t + x_1$, $x_2 = s$, and $x_1 = 0$, what gives the triangle 
    $(0,t). (0,s)$ and $(s-t,s)$, with area $(s-t)^2/2$. We conclude that
    $$
    \pr(|x_1 - x_2| > t) = \frac{(s-t)^2}{s^2} \text{ if } t \in [0,s], 
    $$
    $0$, if $t \ge s$ and $1$ if $t \le 0$. 
    This implies that 
    $$
    \ev[|x_1 - x_2]] = \int_0^{+\infty} \frac{(s-t)^2}{s^2} 1_{\{t \le s\}} \, dt
    = -\frac{(s-t)^3}{3s^2}\bigg|_{0}^s = \frac{s}{3},
    $$
    a simpler prove. 
 \end{proof}

 \item Show that Problem A is equivalent to computing
 \begin{equation*}
  I = \frac{1}{\pi^2 R^4}\int_{0}^{R}\int_{0}^{R}\int_{0}^{2\pi}\int_{0}^{2\pi}\sqrt{r_1^2 + r_2^2 - 2r_1r_2\cos\phi(\theta_1, \theta_2)}\,d\theta_1\,d\theta_2\,dr_1\,dr_2,
 \end{equation*}
 where $\phi(\theta_1, \theta_2)$ is the central angle between $r_1$ and $r_2$.
 
 \textit{Hint:} Draw a picture.
 \item Compute $I$ in closed-form.

 \textit{Hint:} Look up \textit{Crofton's mean value theorem} or \textit{Crofton's formula}. 
 \item Propose a simulation algorithm to approximate $I$.
 Provide point and interval estimates and give theoretical guarantees about them (consistency, coverage, etc).
\end{enumerate}

% 
% \bibliographystyle{apalike}
% \bibliography{refs}

\end{document}          
