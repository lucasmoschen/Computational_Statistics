\documentclass[a4paper,10pt, notitlepage]{report}
\usepackage[utf8]{inputenc}
\usepackage{natbib}
\usepackage{amssymb}
\usepackage{amsmath}
\usepackage[shortlabels]{enumitem}
% \usepackage[portuguese]{babel}


% Title Page
\title{Assignment 0: O Brother, How Far Art Thou?}
\author{Computational Statistics \\ Instructor: Luiz Max de Carvalho}

\begin{document}
\maketitle

\textbf{Hand-in date: 06/10/2021.}

\section*{General guidance}
\begin{itemize}
 \item State and prove all non-trivial mathematical results necessary to substantiate your arguments;
 \item Do not forget to add appropriate scholarly references~\textit{at the end} of the document;
 \item Mathematical expressions also receive punctuation;
 \item Please hand in a single PDF file as your final main document.
 
 Code appendices are welcome,~\textit{in addition} to the main PDF document.
 \end{itemize}

\newpage

\section*{Background}

A large portion of the content of this course is concerned with computing high-dimensional integrals~\textit{via} simulation.
Today you will be introduced to a simple-looking problem with a complicated closed-form solution and one we can approach using simulation.

Suppose you have a disc $C_R$ of radius $R$. 
Take $p = (p_x, p_y)$ and $ q = (q_x, q_y) \in C_R$ two points in the disc.  
Consider the Euclidean distance between $p$  and $q$, $||p-q|| = \sqrt{(p_x-q_x)^2 + (p_y-q_y)^2} = |p-q|$.
\paragraph{Problem A:} What is the \textit{average} distance between pairs of points in $C_R$ if they are picked uniformly at random?

\section*{Questions}

\begin{enumerate}
 \item To start building intuition, let's solve a related but much simpler problem.
 Consider an interval $[0, s]$, with $s>0$ and take $x_1,x_2 \in [0, s]$~\textit{uniformly at random}.
 Show that the average distance between $x_1$ and $x_2$ is $s/3$.
 \item Show that Problem A is equivalent to computing
 \begin{equation*}
  I = \frac{1}{\pi^2 R^4}\int_{0}^{R}\int_{0}^{R}\int_{0}^{2\pi}\int_{0}^{2\pi}\sqrt{r_1^2 + r_2^2 - 2r_1r_2\cos\phi(\theta_1, \theta_2)}r_1r_2\,d\theta_1\,d\theta_2\,dr_1\,dr_2,
 \end{equation*}
 where $\phi(\theta_1, \theta_2)$ is the central angle between $r_1$ and $r_2$.
 
 \textit{Hint:} Draw a picture.
 \item Compute $I$ in closed-form.

 \textit{Hint:} Look up \textit{Crofton's mean value theorem} or \textit{Crofton's formula}. 
 \item Propose a simulation algorithm to approximate $I$.
 Provide point and interval estimates and give theoretical guarantees about them (consistency, coverage, etc).
\end{enumerate}

% 
% \bibliographystyle{apalike}
% \bibliography{refs}

\end{document}          
