\documentclass[a4paper,12pt]{article}

%%%%%%%%%%%%%%%%%%%%%%%%%%%%%%%%%%%%%%%%%%%%%%%%
% Packages
%%%%%%%%%%%%%%%%%%%%%%%%%%%%%%%%%%%%%%%%%%%%%%%%

\usepackage[right=2.5cm, left=2.5cm, top=2.5cm, bottom=2.5cm]{geometry} 
\usepackage[portuguese]{babel}
\usepackage[T1]{fontenc}
\usepackage[utf8]{inputenc}
\usepackage{enumerate}

% no indentation
%\usepackage{setspace}
%\setlength{\parindent}{0in}

\usepackage{graphicx} 
\usepackage{float}
\usepackage{xcolor}
\usepackage{tikz}
\usetikzlibrary{positioning}


\usepackage{mathtools}
\usepackage{amssymb, amsthm}

% headers
\usepackage{fancyhdr}
\usepackage{xurl}
\usepackage{hyperref}



%%%%%%%%%%%%%%%%%%%%%%%%%%%%%%%%%%%%%%%%%%%%%%%%
% Proper definitions
%%%%%%%%%%%%%%%%%%%%%%%%%%%%%%%%%%%%%%%%%%%%%%%%
\newcommand{\R}{\mathbb{R}}
\newcommand{\Q}{\mathbb{Q}}
\newcommand{\Z}{\mathbb{Z}}
\newcommand{\N}{\mathbb{N}}
\newcommand{\tr}{\operatorname{tr}}
\newcommand{\var}{\operatorname{Var}}
\newcommand{\unif}{\operatorname{Unif}}
\newcommand{\ev}{\mathbb{E}}
\newcommand{\pr}{\mathbb{P}}

\newtheorem*{aff}{Afirmação}

\newtheorem{exercise}{Exercício}

\theoremstyle{definition}

%%%%%%%%%%%%%%%%%%%%%%%%%%%%%%%%%%%%%%%%%%%%%%%%
% Header (and Footer)
%%%%%%%%%%%%%%%%%%%%%%%%%%%%%%%%%%%%%%%%%%%%%%%%

\pagestyle{fancy} 
\fancyhf{}

\lhead{\footnotesize CE: Problem sheet 1}
\rhead{\footnotesize Prof. Luiz} 
\cfoot{\footnotesize \thepage} 


\begin{document}

%%%%%%%%%%%%%%%%%%%%%%%%%%%%%%%%%%%%%%%%%%%%%%%%
% Title section of the document
%%%%%%%%%%%%%%%%%%%%%%%%%%%%%%%%%%%%%%%%%%%%%%%%

\thispagestyle{empty} 

\begin{tabular*}{0.95\textwidth}{l @{\extracolsep{\fill}} r} 
    {\large \bf Computational statistics 2021.2} &  \\
    School of Applied Mathematica, Fundação Getulio Vargas &  \\
    Professor Luiz Max de Carvalho  &  \\ 
    \hline \\
\end{tabular*} 
\vspace*{0.3cm} 

\begin{center}
	{\Large \bf Problem sheet 1} 
	\vspace{2mm}
    \\
	{\bf Lucas Machado Moschen}	
\end{center}  
\vspace{0.4cm}

\begin{exercise}
    (Inversion and Rejection)
\end{exercise}

\begin{enumerate}
    \item {\it Let $Y \sim \operatorname{Exp}(\lambda)$ and let $a > 0$. We consider the variable after restricting its support to be $[a, +\infty)$.
    That is, let $X = Y_{|Y \ge a}$, i.e. $X$ has the law of $Y$ conditionally on being in $[a, +\infty)$. Calculate
    $F_X(x)$, the cumulative distribution function of $X$, and
    $F^{-1}_X(u)$, the quantile function of $X$. Describe an algorithm to
    simulate $X$ from $U \sim \unif[0,1]$.}

    If $x \ge a$, we have that
    \begin{equation*}
        \begin{split}
            F_X(x) &= \pr(Y \le x \mid Y \ge a) \\
            &= \frac{\pr(Y \in [a, x])}{\pr(Y \ge a)} \\
            &= \frac{1 - e^{-\lambda x} - (1 - e^{-\lambda a})}{e^{-\lambda a}} \\
            &= 1 - e^{-\lambda(x - a)},
        \end{split}
    \end{equation*}
    otherwise, $F_X(x) = 0$. Let $u = 1 - e^{-\lambda(x - a)}$. Inverting this
    function, we get that 
    $$
    F_X^{-1}(u) = a-\frac{\log(1 - u)}{\lambda}.
    $$

    A simple algorithm is the following 
    \begin{enumerate}[(i)]
        \item Let $U \sim \unif[0,1]$. 
        \item Define $X = F_X^{-1}(U)$. Then $X$ has the desired distribution
        by the inversion method. 
    \end{enumerate}
    
    \item {\it Let $a$ and $b$ be given, with $a < b$. Show that we can simulate $X = Y_
    {|a \le Y \le b}$ from $U \sim \unif[0,1]$ using
    $$
    X= F_Y^{-1}(F_Y(a)(1 - U) + F_Y(b)U),
    $$
    i.e. show that if $X$ is given by the formula above, then $\pr(X \le
    x) = \pr(Y \le x\mid a \le Y \le b)$. Apply
    the formula to simulate an exponential random variable conditioned to
    be greater than a, as in the previous question.}

    Using the properties of the (generalized) inverse and some affine
    transformations, note that
    \begin{equation*}
        \begin{split}
            \pr(X \le x) &= \pr(F_Y^{-1}(F_Y(a)(1 - U) + F_Y(b)U) \le x) \\
            &= \pr(F_Y(a)(1 - U) + F_Y(b)U \le F_Y(x)) \\
            &= \pr(U(F_Y(b) - F_Y(a)) \le F_Y(x) - F_Y(a)) \\
            &= \pr\left(U \le \frac{F_Y(x) - F_Y(a)}{F_Y(b) - F_Y(a)}\right) = \frac{F_Y(x) - F_Y(a)}{F_Y(b) - F_Y(a)}.
        \end{split}
    \end{equation*}
    However, 
    $$\frac{F_Y(x) - F_Y(a)}{F_Y(b) - F_Y(a)} = \frac{\pr(Y \le x) - \pr(Y \le
    a)}{\pr(Y \le b) - \pr(Y \le a)} = \pr(Y \le x \mid Y \in [a,b]),$$
    what concludes that $X$ has the same distribution of $F_Y^{-1}(F_Y(a)(1 -
    U) + F_Y(b)U)$. 

    Taking $b = +\infty$, we can simulate $U \sim \unif[0,1]$ and use 
    $$X = F_Y^{-1}(F_Y(a)(1 - U) + U).$$

    \item {\it Here is a simple algorithm to simulate $X = Y_{|Y > a}$ for $Y
    \sim \operatorname{Exp}(\lambda)$:}

    \begin{enumerate}
        \item {\it Let $Y \sim \operatorname{Exp}(\lambda)$. Simulate $Y = y$.}
        \item {\it If $Y > a$ then stop and return $X = y$, and otherwise, start again at step (a).}
    \end{enumerate}

    {\it Show that this is just a rejection algorithm, by writing the proposal
    and target densities $\pi$ and $q$, as well as the bound $M = \max_x
    \pi(x)/q(x)$. Calculate the expected number of trials to the first
    acceptance. Why is inversion to be preferred for $a \gg 1/\lambda$?}

    The target density $\pi(x) = \frac{d}{dx}F_X(x) = \lambda
    e^{-\lambda(x-a)}1_{\{x \ge a\}}$ is the density of $X$, while the proposal density
    is the exponential $q(x) = \lambda e^{-\lambda x}1_{\{x \ge 0\}}$. Therefore, the bound
    is 
    $$
    M = \sup_{x \ge 0} \frac{\pi(x)}{q(x)} = \sup_{\{x \ge a\}} e^{\lambda a} = e^{\lambda a}. 
    $$
    The probability of accepting $X = y$ is 
    $$
    \alpha(y) = \frac{\pi(y)}{Mq(y)} = \begin{cases}
        0, &\text{ if } y \le a \\
        1, &\text{ if } y > a.
    \end{cases}.
    $$
    This is only the rejection sampling algorithm. Let $N$ be the number os
    trials to the first acceptance.  We already know that $N$ is geometrically
    distributed with parameter $M^{-1} = e^{-\lambda a}$. In our case, this is
    easy to see, because, 
    $$
    \pr(N > n) = \pr(Y \le a)^n = (1 - e^{-\lambda a})^n.
    $$
    We conclude that $\ev[N] = e^{\lambda a}$. When $a \gg 1/\lambda$, we have
    that $\ev[N] \gg e$ and several trials are rejected until a desired
    sample come. In that case, is much simpler to use the inversion method.  
\end{enumerate}

% \bibliographystyle{apalike}
% \bibliography{../stat_comp}

\end{document}          
